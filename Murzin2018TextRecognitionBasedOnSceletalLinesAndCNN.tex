\documentclass[12pt,twoside]{article}
\usepackage{jmlda}
\usepackage[strings]{underscore}

%\NOREVIEWERNOTES
\title
    {Распознавание текста на основе скелетного представления толстых линий и свёрточных сетей}
\author
    {Мурзин~Д.\,А., Местецкий~Л.\,М., Рейер~И.\,А., Стрижов~В.\,В.} % основной список авторов, выводимый в оглавление
\thanks
    {Работа выполнена при финансовой поддержке РФФИ, проект \No\,00-00-00000. 
   Научный руководитель:  Стрижов~В.\,В. 
    Консультант:  Местецкий~Л.\,М.}
\email
    {murzin.da@phystech.edu; mestlm@mail.ru; reyer@forecsys.ru; strijov@phystech.edu}
\organization
    {Московский физико-технический институт}
\abstract
    {В работе рассматривается задача распознавания текста на изображении путём преобразования его в медиальное представление с последующим применением свёрточной нейронной сети для задачи классификации. Данный способ имеет ряд преимуществ по сравнению с классическими дискретными способами распознавания текста. В работе предлагается способ повышения качества распознавания толстых линий за счёт нового способа порождения их описаний. В качестве тестовых данных используются шрифты в растровом представлении.

\bigskip
\textbf{Ключевые слова}: \emph {распознавание текста, непрерывное медиальное представление,
свёрточные нейронные сети}.}
\begin{document}
\maketitle
%\linenumbers
\section{Введение}
Работа посвящена задаче распознавания символов на изображении. Это задача имеет множество применений, от оцифровки старых книг до распознавания рукописного текста.

Существующие методы распознавания текста можно разбить на две группы: «дискретные» и «непрерывные». Дискретные алгоритмы работают с изображением в первоначальном виде, то есть в виде матрицы пикселей. Такой способ обработки изображений близок компьютерам, но не людям, так как мы привыкли различать фигуры и образы, которые являются непрерывными объектами.

С другой стороны, непрерывные алгоритмы построены на использовании таких интуитивных для человека понятий как фигура и форма. Непрерывные алгоритмы устроены примерно следующим образом. Сначала строится непрерывное описание исходного изображения. Это может быть описание границы в виде кривых, либо медиальное представление, то есть набор кривых (скелет) и радиальная функция, которая каждой точке кривой сопоставляет максимальный радиус окружности, лежащей внутри фигуры, с центром в этой точке.

В работе предлагается алгоритм распознавания текста, в котором сначала строится медиальное представление для изображения, с последующим применением свёрточной нейронной сети. Эта сеть состоит из последовательных операций свёртки и уплотнения. В операции свёртки по отдельности рассматривается каждая небольшая часть описания изображения и в ней выделяются характерные паттерны в этой части. Операции уплотнения состоит в уменьшении числа признаков путём замены нескольких частей описания изображения на одну часть, аккумулирующую информацию о найденных паттернах. 

\nocite{*}
\bibliography{Murzin2018TextRecognitionBasedOnSceletalLinesAndCNN}
\bibliographystyle{plain}

% Решение Программного Комитета:
%\ACCEPTNOTE
%\AMENDNOTE
%\REJECTNOTE
\end{document}
