
\documentclass[12pt,twoside]{article}
\usepackage{jmlda}
\usepackage{calligra}
% изменяет \mathcal
% \usepackage{calrsfs}

\newcommand{\x}{\mathbf{x}}
\newtheorem{theorem}{Def2}

%\NOREVIEWERNOTES
\title
{Распознавание текста на основе скелетного представления толстых линий и свёрточных сетей}
\author
{Мурзин~Д.\,А., Местецкий~Л.\,М., Рейер~И.\,А., Стрижов~В.\,В.}
\thanks
{Научный руководитель:  Стрижов~В.\,В. 
	Консультант:  Местецкий~Л.\,М.}
\email
{murzin.da@phystech.edu; mestlm@mail.ru; reyer@forecsys.ru; strijov@phystech.edu}
\organization
{Московский физико-технический институт}
\abstract
{В работе рассматривается задача распознавания текста на изображении путём преобразования его в медиальное представление с последующим применением свёрточной нейронной сети для задачи классификации. Данный способ имеет ряд преимуществ по сравнению с классическими дискретными способами распознавания текста. В работе предлагается способ повышения качества распознавания толстых линий за счёт нового способа порождения их описаний. В качестве тестовых данных используются шрифты в растровом представлении.
	
	\bigskip
	\textbf{Ключевые слова}: \emph {распознавание текста, непрерывное медиальное представление,
		свёрточные нейронные сети}.}
\begin{document}
	\maketitle
	\section{Введение}
	Работа посвящена задаче распознавания символов на изображении. Это задача имеет множество применений, от оцифровки старых книг до распознавания рукописного текста.
	
	Существующие методы распознавания текста можно разбить на две группы: «дискретные» и «непрерывные». Дискретные алгоритмы работают с изображением в первоначальном виде, то есть в виде матрицы пикселей. Такой способ обработки изображений близок компьютерам, но не людям, так как мы привыкли различать фигуры и образы, которые являются непрерывными объектами.
	
	С другой стороны, непрерывные алгоритмы построены на использовании таких интуитивных для человека понятий как фигура и форма. Непрерывные алгоритмы устроены примерно следующим образом. Сначала строится непрерывное описание исходного изображения. Это может быть описание границы в виде кривых, либо медиальное представление, то есть набор кривых (скелет) и радиальная функция, которая каждой точке кривой сопоставляет максимальный радиус окружности, лежащей внутри фигуры, с центром в этой точке.
	
	В работе предлагается алгоритм распознавания текста, в котором сначала строится медиальное представление для изображения, с последующим применением свёрточной нейронной сети. Эта сеть состоит из последовательных операций свёртки и уплотнения. В операции свёртки по отдельности рассматривается каждая небольшая часть описания изображения и в ней выделяются характерные паттерны в этой части. Операции уплотнения состоит в уменьшении числа признаков путём замены нескольких частей описания изображения на одну часть, аккумулирующую информацию о найденных паттернах. 
	
	\section{Постановка задачи}
	В работе решается задача распознования рукописных символов на изображении. Рассматриваются два варианта постановки задачи, <<дискретный>> и <<непрерывный>>, которые отличаются форматом исходных изображений. Опишем постановку задачи, после чего определим форматы изображений для обоих вариантов.
	
	Пусть задано множество символов $\mathcal{S}=\{s_1,\ldots,s_k\}$ и выборка изображений:
	
	$$\mathfrak{D}=\{(\x_i,y_i)|i=1,\ldots,m\}$$
	
	где $\x_i$ является объектом, описывающим $i$-ое изображение, а $y_i \in \mathcal{S}$ --- символом, на нём изображённым. 
	
	Требуется построить алгоритм $f$, решающий задачу классификации изображений, то есть, принимающий описание изображения в том же формате как в исходной выборке и возвращающий список вероятностей $\hat{p}=\{\hat{p}_1,\ldots,\hat{p}_k\}$:
	
	$$f: \x \mapsto (\hat{p_1},\ldots,\hat{p_k})$$
	
	где $\hat{p}_i$ --- предсказание вероятности того что на изображение находится символ $s_i$, $\forall i\ \hat{p}_i \in [0, 1]$, $\hat{p}_1+\ldots+\hat{p}_k=1$. По списку вероятностей можно будет получить предсказание символа на изображении взяв символ с наибольшей вероятностью.
	
	Делаются следующие предположения о выборке:
	
	\begin{itemize}
		\item Каждое изображение содержит ровно один символ, написанный от руки.
		\item Каждый символ на изображении полностью содержится в изображении, причём расстояние между символом и границами изображения строго больше нуля
		\item Каждый символ из множества символов $\mathcal{S}$ встречается достаточно большое число раз в выборке, то есть не существует пар символов $c_1, c_2 \in \mathcal{S}$, таких что символ $c_2$ встречается много больше раз чем символ $c_1$. В идеале равномерное распределение на символах (каждый символ встречается равное число раз).
	\end{itemize}
	
	В качестве функции ошибки для оценки качества алгоритма будем использовать перекрёстную энтропию:
	
	$$H(p, \hat{p}) = -\sum_{i=1}^k p\log{\hat{p}}$$
	
	где $p$ --- истинный вектор вероятностей (все нули кроме одного элемента), $\hat{p}$ --- предсказание вероятностей.
	
	Перейдём к описанию форматов изображений для обоих вариантов постановок.
	
	\section{Постановка задачи (дискретный случай)}
	
	Сначала введём определение множества цветов:
	
	\begin{Def}
		$\mathcal{C}$ --- множество цветов, которые может принимать один пиксель изображения. В работе всегда предполагается $\mathcal{C}=\{0,1\}$, где ноль соответствует белому цвету, а 1 чёрному. Другими возможными вариантами могут быть $\mathcal{C}=\{0,1,\ldots,255\}$ --- оттенки серого и $\mathcal{C}=\{0,1,\ldots,255\}^3$ --- цветовое пространство RGB.
	\end{Def}
	
	В данной постановке описание изображения $\x$ представляет собой матрицу из $h$ строк и $w$ столбцов: $\x_i = [c_{ij}] \in \mathcal{C}^{h\times w}$. Каждый элемент матрицы описывает цвет одного пикселя изображения. Ответ $y_i \in \mathcal{S}$ --- символ, находящийся на изображении $\x_i$.
	
	В работе предлагается использовать базу данных рукописных изображений MNIST \cite{mnist}. В ней каждое изображение имеет размер $28\times28$, а цвета пикселей кодируются числами от 0 до 255 (оттенки серого, 0 --- белый, 255 --- чёрный).
	
	\section{Постановка задачи (непрерывный случай)}
	
	В данной постановке описанием изображения является скелетное представление с заданной на нём радиальной функцией. Введём необходимые определения, в соответствии с \cite{mest2009}:
	
	\begin{Def}
		Фигура --- связная область на плоскости $\mathbb{R}^2$, такая что её граница представляет собой дизъюнктное объединение конечного числа отрезков.
	\end{Def}
	
	\begin{Def}
		Пустой круг фигуры --- круг, полностью содержащийся внутри фигуры.
	\end{Def}
	
	\begin{Def}
		Максимальный пустой круг фигуры --- пустой круг, который не содержится ни в каком другом пустом круге этой фигуры.
	\end{Def}
	
	\begin{Def}
		Скелет фигуры --- связный граф на плоскости, такой что каждая точка каждого ребра графа является центром максимального пустого круга.
	\end{Def}
	
	\begin{Def}
		Радиальная функция для скелетного представления --- функция, которая каждой точке скелетного представления сопоставляет радиус максимального круга с центром в этой точке.
	\end{Def}
	
	\begin{Def}
		Медиальное представление фигуры --- скелет фигуры с соответствующей медиальной функцией.
	\end{Def}
	
	В работе предлагается использовать выборку, в которой медиальное представление имеет следующий вид: скелет задан в виде графа, радиальная функция задана на каждой вершине этого графа, а значение радиальной функции на рёбрах определяется как взвешенное среднее радиальной функции на концах ребра. 
	
	Также, дополнительно, каждая вершина имеет степень от одного до трёх.
	
	\section{Базовый алгоритм}
	
	В качестве базового алгоритма используется свёрточная нейронная сеть для задачи в дискретной постановке. Предлагается использовать следующую структуру сети:
	
	$$\text{INPUT} \rightarrow [[\text{CONV} \rightarrow \text{RELU}]\times2 \rightarrow \text{POOL}]\times3 \rightarrow \text{FC}$$
	
	\begin{itemize}
		\item INPUT --- входной слой, имеет размеры $28\times28\times1$
		\item CONV --- слой свёртки. Фильтры имеют размер $3\times3$. Также используется увеличение пространственных размеров на 2 в каждой размерности предыдущего слоя путём дополнения одинарной линией из нулей с каждой стороны.
		\item RELU --- слой активации. Используется функция $f(x)=\max(0,x)$
		\item POOL --- слой пулинга. Каждая группа пикселей $2\times2$ уплотняется в один пиксель, путём взятия максимума.
		\item FC --- полносвязный слой. 
	\end{itemize}
	
	Обучение сети будет осуществляться методом обратного распространения ошибки.
	
	\nocite{*}
	\bibliography{Murzin2018TextRecognitionUsingSkeletonRepresentationAndCNN}
	\bibliographystyle{unsrt}
	
	% Решение Программного Комитета:
	%\ACCEPTNOTE
	%\AMENDNOTE
	%\REJECTNOTE
\end{document}
